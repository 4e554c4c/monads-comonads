\documentclass{amsart}
\usepackage{enumitem}
\usepackage{hyperref}
\usepackage{blindtext}
\usepackage{amsmath}
\usepackage{mathtools}
\usepackage{amsmath}
\usepackage{amsthm}
\usepackage{amssymb}
\usepackage{siunitx}
\usepackage[super]{nth}
\usepackage{relsize}
\usepackage{tikz-cd}
\usepackage{manfnt}
\usepackage{cleveref}
\usepackage{minted}
%\usepackage{txfonts} %\multimapinv, etc
\usepackage[
  backend=biber,
  style=numeric,
  sorting=none,
]{biblatex}
\usepackage{tangle}
\tgColours{F5A3A3,F5CCA3,F5F5A3,CCF5A3,A3F5A3,A3F5CC,A3F5F5,A3CCF5,A3A3F5,CCA3F5,F5A3F5,F5A3CC}


\addbibresource{citations.bib}

\newtheorem{lemma}[equation]{Lemma}
\newtheorem{corollary}[equation]{Corollary}
\newtheorem{proposition}[equation]{Proposition}
\theoremstyle{remark}
\newtheorem*{answer}{Answer}

\DeclarePairedDelimiter\abs{\lvert}{\rvert}
\DeclarePairedDelimiter\norm{\lVert}{\rVert}
\DeclarePairedDelimiter\ceil{\lceil}{\rceil}
\DeclarePairedDelimiter\floor{\lfloor}{\rfloor}
\DeclarePairedDelimiter\inner{\langle}{\rangle}
\DeclareMathOperator{\sgn}{sgn}
\makeatletter
\newcommand{\skipitem}[1][1]{%
  \addtocounter{\@enumctr}{#1}%
}
\newcommand{\ppd}[2][{}]{\ensuremath{\frac{\partial #1}{\partial #2}}}
\newcommand{\multimapinv}{\mathbin{\mathpalette\@multimapinv\relax}}
\newcommand{\@multimapinv}[2]{\rotatebox[origin=c]{180}{$#1\multimap$}}
\makeatother


%\renewcommand{\thefootnote}{\fnsymbol{footnote}}

\title{Preliminary Research Report}
\author{Calvin Santiago Lee}
\begin{document}
\maketitle
\tableofcontents
\section{Introduction}

This research project concerns the paper ``Interaction laws of Monads and Comonads'' by Katsumata, Rivas, and Uustalu\cite{monads-comonads}. This paper investigates the relation between effectual computations  and environments, modeled as monads and comonads respectively in some base category. The paper first defines this relationship, and then investigates generating a canonical (co)monad given a (co)monad $T$.

This project involves formalizing several major claims and examples from the paper in a theorem prover. Specifically, the project will require defining the monoidal category of functor-functor (and monad-comonad) interaction laws, and the theory of comonad duals.

\section{Methodology}
We have chosen to formalize the paper in the Agda\cite{agda} theorem prover. As the paper depends heavily on category theory, the choice of category theory library heavily influences the style of the formalization. The agda-categories\cite{agda:categories} library by Hu and Carette was selected due to a detailed formalization of many category theoretic notions important in the paper.

\section{Implementation Decisions}
As opposed to the approach taken in the paper, and in previous formalizations, the decision has been made to formalize the paper using the language of Bicategories\cite{nlab:bicategory}. For example, in the paper a functor-functor interaction law between two functors $F$ and $G$ is given as a family of maps 
\[\phi_{X,Y}\colon F X \times GY \to X \times Y\]
with a naturality condition in $X$ and $Y$.
In the formalization, this is instead given by a natural transformation \[\phi \colon F \otimes G \Rightarrow -\otimes -,\]
i.e. a morphism in the category of bifunctors \(C^{C\times C}\) for the underlying category $C$.

This design additionally lends itself to a graphical calculus---that of string diagrams--- which aid in construction and proof. For example the above definition of a functor-functor interaction law may be denoted as follows, where the category C is denoted as \(
% https://varkor.github.io/tangle/?t=W1tbMV1dLFtbWzAsW11dXV1d&c=F5A3A3,F5CCA3,F5F5A3,CCF5A3,A3F5A3,A3F5CC,A3F5F5,A3CCF5,A3A3F5,CCA3F5,F5A3F5,F5A3CC
\begin{tangle}{(1,1)}[trim x,trim y]
	\tgBlank{(0,0)}{\tgColour0}
\end{tangle}
\) and the category $C\times C$ as 
\(
\begin{tangle}{(1,1)}[trim x,trim y]
	\tgBlank{(0,0)}{\tgColour7}
\end{tangle}
\):
\[
% https://varkor.github.io/tangle/?t=W1tbMSw4LDhdXSxbW1sxLFsxLDEsMSwwXV0sWzIsWzNdXV1dLFtbMCwwLjUsMC41LFsiXFxwaGkiXV1dLFtbMCwwLDMsIlxcb3RpbWVzIl0sWzAsMCwxLCJcXG90aW1lcyJdLFsxLDAsMSwiRlxcdGltZXMgRyJdXV0=&c=F5A3A3,F5CCA3,F5F5A3,CCF5A3,A3F5A3,A3F5CC,A3F5F5,A3CCF5,A3A3F5,CCA3F5,F5A3F5,F5A3CC
\begin{tangle}{(2,1)}
	\tgBorderA{(0,0)}{\tgColour0}{\tgColour7}{\tgColour7}{\tgColour0}
	\tgBorder{(0,0)}{0}{1}{0}{0}
	\tgBorderC{(1,0)}{2}{\tgColour7}{\tgColour7}
	\tgCell{(0,0)}{\phi}
	\tgAxisLabel{(0.5,0)}{south}{\otimes}
	\tgAxisLabel{(0.5,1)}{north}{\otimes}
	\tgAxisLabel{(1.5,1)}{north}{F\times G}
\end{tangle}.
\]
Other more complicated concepts are represented well in this graphical notation. For example, the functor-functor interaction law morphism between two functor-functor interaction laws $\phi$ and $\psi$ consists of two natural transformations $f\colon F \Rightarrow F'$ and $g\colon G' \Rightarrow G$ as well as the coherence law

\[
% https://varkor.github.io/tangle/?t=W1tbMSw4LDhdXSxbW1swLFtdXSxbMSxbMSwxLDEsMF1dLFsyLFszXV0sWzAsW11dXSxbWzAsW11dLFsxLFsxLDAsMSwwXV0sWzEsWzEsMCwxLDBdXSxbMCxbXV1dXSxbWzAsMS41LDAuNSxbIlxccHNpIl1dLFswLDIuNSwxLFsiXFxtYXRocm17aWR9XFx0aW1lcyBnIiwxXV1dLFtbMSwwLDMsIlxcb3RpbWVzIl0sWzEsMSwxLCJcXG90aW1lcyJdLFsyLDEsMSwiRidcXHRpbWVzIEciXV1d&c=F5A3A3,F5CCA3,F5F5A3,CCF5A3,A3F5A3,A3F5CC,A3F5F5,A3CCF5,A3A3F5,CCA3F5,F5A3F5,F5A3CC
\begin{tangle}{(4,2)}[trim x]
	\tgBlank{(0,0)}{\tgColour0}
	\tgBorderA{(1,0)}{\tgColour0}{\tgColour7}{\tgColour7}{\tgColour0}
	\tgBorder{(1,0)}{0}{1}{0}{0}
	\tgBorderC{(2,0)}{2}{\tgColour7}{\tgColour7}
	\tgBlank{(3,0)}{\tgColour7}
	\tgBlank{(0,1)}{\tgColour0}
	\tgBorderA{(1,1)}{\tgColour0}{\tgColour7}{\tgColour7}{\tgColour0}
	\tgBorderA{(2,1)}{\tgColour7}{\tgColour7}{\tgColour7}{\tgColour7}
	\tgBorder{(2,1)}{1}{0}{1}{0}
	\tgBlank{(3,1)}{\tgColour7}
	\tgCell{(1,0)}{\phi}
	\tgCell[(1,0)]{(2,0.5)}{\mathrm{id}\times g}
	\tgAxisLabel{(1.5,0)}{south}{\otimes}
	\tgAxisLabel{(1.5,2)}{north}{\otimes}
	\tgAxisLabel{(2.5,2)}{north}{F'\times G}
\end{tangle}
=
% https://varkor.github.io/tangle/?t=W1tbMSw4LDhdXSxbW1swLFtdXSxbMSxbMSwxLDEsMF1dLFsyLFszXV0sWzAsW11dXSxbWzAsW11dLFsxLFsxLDAsMSwwXV0sWzEsWzEsMCwxLDBdXSxbMCxbXV1dXSxbWzAsMS41LDAuNSxbIlxccHNpIl1dLFswLDIuNSwxLFsiIiwxXV1dLFtbMSwwLDMsIlxcb3RpbWVzIl0sWzEsMSwxLCJcXG90aW1lcyJdLFsyLDEsMSwiRidcXHRpbWVzIEciXV1d&c=F5A3A3,F5CCA3,F5F5A3,CCF5A3,A3F5A3,A3F5CC,A3F5F5,A3CCF5,A3A3F5,CCA3F5,F5A3F5,F5A3CC
\begin{tangle}{(4,2)}[trim x]
	\tgBlank{(0,0)}{\tgColour0}
	\tgBorderA{(1,0)}{\tgColour0}{\tgColour7}{\tgColour7}{\tgColour0}
	\tgBorder{(1,0)}{0}{1}{0}{0}
	\tgBorderC{(2,0)}{2}{\tgColour7}{\tgColour7}
	\tgBlank{(3,0)}{\tgColour7}
	\tgBlank{(0,1)}{\tgColour0}
	\tgBorderA{(1,1)}{\tgColour0}{\tgColour7}{\tgColour7}{\tgColour0}
	\tgBorderA{(2,1)}{\tgColour7}{\tgColour7}{\tgColour7}{\tgColour7}
	\tgBorder{(2,1)}{1}{0}{1}{0}
	\tgBlank{(3,1)}{\tgColour7}
	\tgCell{(1,0)}{\psi}
	\tgCell[(1,0)]{(2,0.5)}{f\times \mathrm{id}}
	\tgAxisLabel{(1.5,0)}{south}{\otimes}
	\tgAxisLabel{(1.5,2)}{north}{\otimes}
	\tgAxisLabel{(2.5,2)}{north}{F'\times G}
\end{tangle}.
\]

%This definition has merits by being more readily composable than notions which separate the data of a natural transformation and its naturality conditions (as described below).

\section{Initial Results}

The majority of Section 2.1 of the paper has formalized, including examples. In particular, the definition of interaction laws and morphisms thereof have been formalized. The categorical structure of $\mathbf{IL}$ has also been formalized. Both are verified by Agda with the \verb|--safe| flag which ensures that no theorems or metavariables have been missed. 

The bifunctor which defines the monoidal product in $\mathbf{IL}$ has also been defined, and properties proved. However, the remainder of the monoidal structure, such as unitor, associator, triangle and pentagon identites have \emph{not} been proved. Neither are they mentioned in the paper.

As a test, several examples have been formalized. First, the interaction-law between the ``reader'' monad $F=[A, -]$ and ``co-reader`` comonad $G=A\otimes -$ has been formalized for any $A : C_0$ for a general closed-monoidal category $C$.

Next, in a biclosed monoidal category\footnote{biclosed monoidal categories have not been defined in agda-categories\cite{agda:categories}, so the necessary machinery was added in the formalization.}
the interaction between the generalized reader/coreader $F=-\multimapinv A$, $G=A\otimes -$ and writer/cowriter $J=-\otimes A$, $K=A\multimap -$ have been formalized. Finally, we formalized that the monoidal product of these two interaction laws gives an interaction law between state $F\circ J$ and costate $G\circ K$ by definition.


\section{Additional Considerations}

\subsection{Alternative theorem provers}

Agda was chosen due to due to a variety of relevant libraries and powerful theorem proving techniques. However, it is not the only option. The Lean theorem prover (version 4), as described in Ullrich's dissertation\cite{ullrich-dissertation} is a new theorem prover with a mature and extensive associated mathematical formalization mathlib. Lean is successful in proving theorems in a type-theoretic foundations using classical techniques. In particular, classical mathematical axioms such as UIP, quotient soundness, propositional extensionality and classical choice are assumed in the mathlib repository.

Although these axioms are often useful in Lean to simplify proofs, they inherently exclude additional mathematical foundations. For example, the presence of UIP in the type theory does not allow Lean proofs to be utilized in homotopy type theory, which is inconsistent with the principle\cite{hottbook}. Furthermore, the axiom of propositional extensionally, which is heavily used in mathlib to model sets, creates a failure in normalization as shown by Abel and Coquand\cite{lean-normalization}.

\subsection{Alternative category theory libraries}

As described above, the Agda-categories\cite{agda:categories} library was selected due to a large and detailed formalization of category theory. Agda-categories makes several opinionated design decisions. First, it embraces a limited axiomatic system with proof-relevance, especially when relating elements of hom-sets. Special care is taken throughout the library to insure the involute nature of the categorical dual. This requires $\eta$-equality on every record (see next section) and the doubling of many axioms.
Despite this choice of library, many other formalizations exist.

Unimath\cite{agda:unimath} is a large formalization of mathematics in univalant Agda by Egbert Rijke et al., primarily focusing on group-theory. It contains many important definitions in category theory. However, like Lean formalizations, it lacks computable realization due to postulating univalence. Furthermore, it attempts to use a minimal number of agda-features, which makes use of the formalization difficult.

The 1lab\cite{agda:1lab} is an implementation of mathematics in cubical Agda (see Vezzosi, M\"{o}rtberg and Abel\cite{agda:cubical}). While not as developed as either unimath or agda-categories, the 1lab offers an alternative formalization which lends itself better to certain issues (see the following section).

\subsection{$\eta$-equality and metavariable inference}
$\eta$-expansion in Agda for a record
\begin{minted}{agda}
record R : Set where
  field
    a : A
    b : B
\end{minted}
is the identification of an element \mintinline{agda}{r : R} with \mintinline{agda}{record { a = R.a r ; b = R.b r }}.


\section{Future Work}
\blindtext

\printbibliography
\end{document}
