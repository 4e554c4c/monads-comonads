\documentclass{amsart}
\usepackage{enumitem}
\usepackage{hyperref}
\usepackage{blindtext}
\usepackage{amsmath}
\usepackage{mathtools}
\usepackage{amsmath}
\usepackage{amsthm}
\usepackage{amssymb}
\usepackage{siunitx}
\usepackage[super]{nth}
\usepackage{relsize}
\usepackage{tikz-cd}
\usepackage{manfnt}
\usepackage{cleveref}
%\usepackage{txfonts} %\multimapinv, etc
\usepackage[
  backend=biber,
  style=numeric,
]{biblatex}
\usepackage{tangle}
\tgColours{F5A3A3,F5CCA3,F5F5A3,CCF5A3,A3F5A3,A3F5CC,A3F5F5,A3CCF5,A3A3F5,CCA3F5,F5A3F5,F5A3CC}

\newcommand{\multimapinv}{\mathbin{\mathpalette\rotmmap\relax}}
\newcommand{\rotmmap}[2]{\rotatebox[origin=c]{180}{$#1\multimap$}}


\addbibresource{citations.bib}

\newtheorem{lemma}[equation]{Lemma}
\newtheorem{corollary}[equation]{Corollary}
\newtheorem{proposition}[equation]{Proposition}
\theoremstyle{remark}
\newtheorem*{answer}{Answer}

\DeclarePairedDelimiter\abs{\lvert}{\rvert}
\DeclarePairedDelimiter\norm{\lVert}{\rVert}
\DeclarePairedDelimiter\ceil{\lceil}{\rceil}
\DeclarePairedDelimiter\floor{\lfloor}{\rfloor}
\DeclarePairedDelimiter\inner{\langle}{\rangle}
\DeclareMathOperator{\sgn}{sgn}
\makeatletter
\newcommand{\skipitem}[1][1]{%
  \addtocounter{\@enumctr}{#1}%
}
\newcommand{\ppd}[2][{}]{\ensuremath{\frac{\partial #1}{\partial #2}}}

%\renewcommand{\thefootnote}{\fnsymbol{footnote}}

\title{Preliminary Research Report}
\author{Calvin Santiago Lee}
\begin{document}
\maketitle
\tableofcontents
\section{Introduction}

This research project concerns the paper ``Interaction laws of Monads and Comonads'' by Katsumata, Rivas, and Uustalu\cite{monads-comonads}. This paper investigates the relation between effectual computations  and environments, modeled as monads and comonads respectively in some base category. The paper first defines this relationship, and then investigates generating a canonical (co)monad given a (co)monad $T$.

This project involves formalizing several major claims and examples from the paper in a theorem prover. Specifically, the project will require defining the monoidal category of functor-functor (and monad-comonad) interaction laws, and the theory of comonad duals.

\section{Methodology}
We have chosen to formalize the paper in the Agda\cite{agda} theorem prover due to a variety of relevant libraries and powerful theorem proving techniques. As the paper depends heavily on category theory, the choice of category theory library heavily influences the style of the formalization. The agda-categories\cite{agda:categories} library by Hu and Carette was selected due to a detailed formalization of many category theoretic notions important in the paper.

\section{Implementation Decisions}
As opposed to the approach taken in the paper, and in previous formalizations, the decision has been made to formalize the paper using the language of Bicategories\cite{nlab:bicategory}. For example, in the paper a functor-functor interaction law between two functors $F$ and $G$ is given as a family of maps 
\[\phi_{X,Y}\colon F X \times GY \to X \times Y\]
with a naturality condition in $X$ and $Y$.
In the formalization, this is instead given by a natural transformation \[\phi \colon F \otimes G \Rightarrow -\otimes -,\]
i.e. a morphism in the category of bifunctors \(C^{C\times C}\) for the underlying category $C$.

This design additionally lends itself to a graphical calculus---that of string diagrams--- which aid in construction and proof. For example the above definition of a functor-functor interaction law may be denoted as follows, where the category C is denoted as \(
% https://varkor.github.io/tangle/?t=W1tbMV1dLFtbWzAsW11dXV1d&c=F5A3A3,F5CCA3,F5F5A3,CCF5A3,A3F5A3,A3F5CC,A3F5F5,A3CCF5,A3A3F5,CCA3F5,F5A3F5,F5A3CC
\begin{tangle}{(1,1)}[trim x,trim y]
	\tgBlank{(0,0)}{\tgColour0}
\end{tangle}
\) and the category $C\times C$ as 
\(
\begin{tangle}{(1,1)}[trim x,trim y]
	\tgBlank{(0,0)}{\tgColour7}
\end{tangle}
\):
\[
% https://varkor.github.io/tangle/?t=W1tbMSw4LDhdXSxbW1sxLFsxLDEsMSwwXV0sWzIsWzNdXV1dLFtbMCwwLjUsMC41LFsiXFxwaGkiXV1dLFtbMCwwLDMsIlxcb3RpbWVzIl0sWzAsMCwxLCJcXG90aW1lcyJdLFsxLDAsMSwiRlxcdGltZXMgRyJdXV0=&c=F5A3A3,F5CCA3,F5F5A3,CCF5A3,A3F5A3,A3F5CC,A3F5F5,A3CCF5,A3A3F5,CCA3F5,F5A3F5,F5A3CC
\begin{tangle}{(2,1)}
	\tgBorderA{(0,0)}{\tgColour0}{\tgColour7}{\tgColour7}{\tgColour0}
	\tgBorder{(0,0)}{0}{1}{0}{0}
	\tgBorderC{(1,0)}{2}{\tgColour7}{\tgColour7}
	\tgCell{(0,0)}{\phi}
	\tgAxisLabel{(0.5,0)}{south}{\otimes}
	\tgAxisLabel{(0.5,1)}{north}{\otimes}
	\tgAxisLabel{(1.5,1)}{north}{F\times G}
\end{tangle}.
\]
Other more complicated concepts are represented well in this graphical notation. For example, the functor-functor interaction law morphism between two functor-functor interaction laws $\phi$ and $\psi$ consists of two natural transformations $f\colon F \Rightarrow F'$ and $g\colon G' \Rightarrow G$ as well as the coherence law

\[
% https://varkor.github.io/tangle/?t=W1tbMSw4LDhdXSxbW1swLFtdXSxbMSxbMSwxLDEsMF1dLFsyLFszXV0sWzAsW11dXSxbWzAsW11dLFsxLFsxLDAsMSwwXV0sWzEsWzEsMCwxLDBdXSxbMCxbXV1dXSxbWzAsMS41LDAuNSxbIlxccHNpIl1dLFswLDIuNSwxLFsiXFxtYXRocm17aWR9XFx0aW1lcyBnIiwxXV1dLFtbMSwwLDMsIlxcb3RpbWVzIl0sWzEsMSwxLCJcXG90aW1lcyJdLFsyLDEsMSwiRidcXHRpbWVzIEciXV1d&c=F5A3A3,F5CCA3,F5F5A3,CCF5A3,A3F5A3,A3F5CC,A3F5F5,A3CCF5,A3A3F5,CCA3F5,F5A3F5,F5A3CC
\begin{tangle}{(4,2)}[trim x]
	\tgBlank{(0,0)}{\tgColour0}
	\tgBorderA{(1,0)}{\tgColour0}{\tgColour7}{\tgColour7}{\tgColour0}
	\tgBorder{(1,0)}{0}{1}{0}{0}
	\tgBorderC{(2,0)}{2}{\tgColour7}{\tgColour7}
	\tgBlank{(3,0)}{\tgColour7}
	\tgBlank{(0,1)}{\tgColour0}
	\tgBorderA{(1,1)}{\tgColour0}{\tgColour7}{\tgColour7}{\tgColour0}
	\tgBorderA{(2,1)}{\tgColour7}{\tgColour7}{\tgColour7}{\tgColour7}
	\tgBorder{(2,1)}{1}{0}{1}{0}
	\tgBlank{(3,1)}{\tgColour7}
	\tgCell{(1,0)}{\phi}
	\tgCell[(1,0)]{(2,0.5)}{\mathrm{id}\times g}
	\tgAxisLabel{(1.5,0)}{south}{\otimes}
	\tgAxisLabel{(1.5,2)}{north}{\otimes}
	\tgAxisLabel{(2.5,2)}{north}{F'\times G}
\end{tangle}
=
% https://varkor.github.io/tangle/?t=W1tbMSw4LDhdXSxbW1swLFtdXSxbMSxbMSwxLDEsMF1dLFsyLFszXV0sWzAsW11dXSxbWzAsW11dLFsxLFsxLDAsMSwwXV0sWzEsWzEsMCwxLDBdXSxbMCxbXV1dXSxbWzAsMS41LDAuNSxbIlxccHNpIl1dLFswLDIuNSwxLFsiIiwxXV1dLFtbMSwwLDMsIlxcb3RpbWVzIl0sWzEsMSwxLCJcXG90aW1lcyJdLFsyLDEsMSwiRidcXHRpbWVzIEciXV1d&c=F5A3A3,F5CCA3,F5F5A3,CCF5A3,A3F5A3,A3F5CC,A3F5F5,A3CCF5,A3A3F5,CCA3F5,F5A3F5,F5A3CC
\begin{tangle}{(4,2)}[trim x]
	\tgBlank{(0,0)}{\tgColour0}
	\tgBorderA{(1,0)}{\tgColour0}{\tgColour7}{\tgColour7}{\tgColour0}
	\tgBorder{(1,0)}{0}{1}{0}{0}
	\tgBorderC{(2,0)}{2}{\tgColour7}{\tgColour7}
	\tgBlank{(3,0)}{\tgColour7}
	\tgBlank{(0,1)}{\tgColour0}
	\tgBorderA{(1,1)}{\tgColour0}{\tgColour7}{\tgColour7}{\tgColour0}
	\tgBorderA{(2,1)}{\tgColour7}{\tgColour7}{\tgColour7}{\tgColour7}
	\tgBorder{(2,1)}{1}{0}{1}{0}
	\tgBlank{(3,1)}{\tgColour7}
	\tgCell{(1,0)}{\psi}
	\tgCell[(1,0)]{(2,0.5)}{f\times \mathrm{id}}
	\tgAxisLabel{(1.5,0)}{south}{\otimes}
	\tgAxisLabel{(1.5,2)}{north}{\otimes}
	\tgAxisLabel{(2.5,2)}{north}{F'\times G}
\end{tangle}.
\]

%This definition has merits by being more readily composable than notions which separate the data of a natural transformation and its naturality conditions (as described below).

\section{Initial Results}

The majority of Section 2.1 of the paper has formalized, including examples. In particular, the definition of interaction laws and morphisms thereof have been formalized. The categorical structure of $\mathbf{IL}$ has also been formalized. Both are verified by Agda with the \verb|--safe| flag which ensures that no theorems or metavariables have been missed. 

The bifunctor which defines the monoidal product in $\mathbf{IL}$ has also been defined, and properties proved. However, the remainder of the monoidal structure, such as unitor, associator, triangle and pentagon identites have \emph{not} been proved. Neither are they mentioned in the paper.

As a test, several examples have been formalized. First, the interaction-law between the ``reader'' monad $F=[A, -]$ and ``co-reader`` comonad $G=A\otimes -$ has been formalized for any $A : C_0$ for a general closed-monoidal category $C$.

Next, in a biclosed monoidal category\footnote{biclosed monoidal categories have not been defined in agda-categories\cite{agda:categories}, so the necessary machinery was added in the formalization.}
the interaction between the generalized reader/coreader $F=-\multimapinv A$, $G=A\otimes -$ and writer/cowriter $J=-\otimes A$, $K=A\multimap -$ have been formalized. Finally, we formalized that the monoidal product of these two interaction laws gives an interaction law between state $F\circ J$ and costate $G\circ K$ by definition.


\section{Additional Considerations}

\subsection{Alternative theorem provers}

\subsection{Alternative category theory libraries}

\subsection{$\eta$-equality and metavariable inference}

\section{Future Work}

\printbibliography
\end{document}
